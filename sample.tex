\documentclass[11pt]{ltjsarticle}

\usepackage{tdu-ec-report-style/tdu-ec-report}

\title{サンプル}

\graphicspath{ {./figs/} }

\begin{document}
\section{実験日・気温・湿度・天候}
\subsection{実験日}
2025年5月20日

\subsection{気温}
25 ℃

\subsection{湿度}
20 \%

\subsection{天候}
晴天

\section{目的}
\TeX を基にして機能強化した \LaTeX \cite{LaTeX}を用いて,レポートを記述する基礎を学ぶ.

\section{使用機器}
本実験で使用した機器名とメーカー名及び型番とその数量を表\ref{used}に示す.

\begin{table}[!htbp]
  \centering
  \caption{使用機器名とそのメーカー名及び型番と数量}
  \label{used}
  \begin{tblr}{
      hlines = {dash=solid},
      vlines = {dash=solid},
      rows = {halign = c},
    }
    機器名      & メーカー名   & 型番      & 数量 \\
    シリアル送受信機 & 情報通信実験室 & ---     & 2  \\
    オシロスコープ  & IWATSU  & DS5105B & 1  \\
  \end{tblr}
\end{table}

\pagebreak

\section{実験1}
\LaTeX で生成する文中に図を表示する方法を学ぶ.

\subsection{実験方法}
このソースコードに図を挿入した.

\subsection{実験結果}
挿入した結果を図\ref{fig:example-image}に示す.
\begin{figure}[H]
  \centering
  \includegraphics[width=14cm]{example-image}
  \caption{図の挿入例}
  \label{fig:example-image}
\end{figure}

\pagebreak

\section{検討事項}

\pagebreak

\section{吟味}

\pagebreak

\begin{thebibliography}{9}
  \bibitem{LaTeX}
  奥村晴彦,黒木裕介:”[改訂第9版] \LaTeX 美文書作成入門”,技術評論社,p.1,2023.
\end{thebibliography}

\end{document}
